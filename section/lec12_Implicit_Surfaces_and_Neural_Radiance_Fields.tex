\section{Implicit Surfaces and Neural Radiance Fields}

3D representations: Voxels (Discretization of 3D space into grid $\mathcal{O}(n^3)$, low res),Points cloud (no topology/connectivity), meshes (vertices and faces, mixed res, self-inters, not general), Implicit (level set \(S = \{x: f(x) = 0\}\) where $f$ continuous).

\subsection*{Neural Implicit Representations (NIR)}

\textbf{Occupancy nets} \(f_{\theta}: \mathbb{R}^{3} \times \mathcal{X} \rightarrow[0,1]\), 3D location and cond $\rightarrow$ occupancy; \textbf{Signed dist field: DeepSDF} \(f_{\theta}: \mathbb{R}^{3} \times \mathcal{X} \rightarrow \mathbb{R}\)

\subsection*{Supervise NIR from other reprs:}

(1) \textbf{Watertight meshes} sample random points then cross-entropy loss.

(2) \textbf{Points cloud: Implicit Geom Regulariz (IGR)} unordered and hard, \(\chi=\left\{x_{i}\right\}_{i \in I} \subset \mathbb{R}^{3}\)), learn \(f_{\theta}(x)\) is approximately the signed distance function to a plausible surface \(M\) defined by \(\chi\).
% 
Continuous Shortest Path (Eikonal PDE), when \(\|\nabla f(x)\|= 1\), learned \(f\) is $\approx$ the distance to the suface.
% 
Loss: \(\mathcal{L}(\theta)=\sum_{i \in I}\left|f_{\theta}\left(x_{i}\right)\right|^{2}+\lambda \mathbb{E}_{x}\left(\left\|\nabla_{x} f_{\theta}(x)\right\|-1\right)^{2}\).

(3) \textbf{Img/monocular vid: Differentiable Volumetric Rendering (DVR)}
% 
NN output texture/color \(t_\theta(p)\) and occupancy \(f_\theta(p) \in [0, 1]\).
% 
\textbf{Forward}
1. given observation position \(\mathbf{r}_0\) and relative pixel \(\to\) direction of the ray \(\mathbf{w}\),
2. find the solution \(d\) for surface, \(f_\theta (\mathbf{p} = \mathbf{r}_0 + d \mathbf{w}) = 0\),
3. use \(t_\theta(\mathbf{p} = \mathbf{r}_0 + d \mathbf{w})\) for texture and color. Secant method for linesearch of zero point \(x_{2}=x_{1}-f\left(x_{1}\right) \frac{x_{1}-x_{0}}{f\left(x_{1}\right)-f\left(x_{0}\right)}\)

\textbf{Backward}
1. Loss: differnce w.r.t images, \(\mathcal{L}(\hat{\mathbf{I}}, \mathbf{I})=\sum_{\mathbf{u}}\left\|\hat{\mathbf{I}}_{\mathbf{u}}-\mathbf{I}_{\mathbf{u}}\right\|\),
% 
\(\frac{\partial \mathcal{L}}{\partial \theta}=\sum_{\mathbf{u}} \frac{\partial \mathcal{L}}{\partial \hat{\mathbf{I}}_{\mathbf{u}}} \cdot \frac{\partial \hat{\mathbf{I}}_{\mathbf{u}}}{\partial \theta},
% 
\frac{\partial \hat{\mathbf{I}}_{\mathbf{u}}}{\partial \theta}=\frac{\partial \mathrm{t}_{\theta}(\widehat{\mathbf{p}})}{\partial \theta}+\frac{\partial \mathrm{t}_{\theta}(\widehat{\mathbf{p}})}{\partial \widehat{\mathbf{p}}} \cdot \frac{\partial \widehat{\mathbf{p}}}{\partial \theta}\),
2. From \(f_{\theta}(\widehat{\mathbf{p}})=0\) and \(\widehat{\mathbf{p}}=\mathbf{r}_{0}+\hat{d}(\theta) \mathbf{w}\) \(\to\) implicit gradient \(\frac{\partial \widehat{\mathbf{p}}}{\partial \theta}=-\mathbf{w}\left(\frac{\partial f_{\theta}(\widehat{\mathbf{p}})}{\partial \widehat{\mathbf{p}}} \cdot \mathbf{w}\right)^{-1} \frac{\partial f_{\theta}(\widehat{\mathbf{p}})}{\partial \theta}\).


\subsection*{Neural Radiance Field (NeRF)}
\textbf{Motivation} Surfaces are good, but scenes are more complex, need opacity.

\textbf{NN} \(F_\theta(x,y, z, \theta, \phi) = (r,g,b, \sigma)\), \(\sigma\) density.

\((\theta, \phi)\) view direction, added at late stage of network, to avoid trivial sol.

% , cuz most of the texture is not view dependent, othervise, will fall into trivial solution of sphere with complex texture. Density (geometry) is independent of viewing direction. Viewing direction only applied at a later layer, which limits the viewdependent effects and thus encourages detailed geometry.

\textbf{Volume Rendering}
% 
\(\alpha=1-\mathrm{e}^{\left(-\sigma_{i} \delta_{i}\right)}, \delta_{i}=t_{i+1}-t_{i}\), Transmittance \(T_{i}=\prod_{j=1}^{i-1}\left(1-\alpha_{j}\right)\), Final ray color \(c=\sum_{i=1}^{N} {T_{i} \alpha_{i}} c_{i}\).

\textbf{Train} \(\min _{\theta} \sum_{i}\left\|\operatorname{render}_{i}\left(F_{\theta}\right)-I_{i}\right\|^{2}\), sampling efficiency is a big issue.

\textbf{Positional encoding} pass low-dim \(x,y,z\) coordinates via fixed positional encoding controlled by \(L\) or random Fourier features of varying frequencies, instead of directly use \((x,y,z)\), to model higher freqs.
% 
\(\mathsf{PE} = \text{cat}[\cos k\mathbf{v}, \sin k\mathbf{v}]_{k=1}^{2^L}\), 
% 
or \(\gamma(\mathbf{v})=[\cos (\mathbf{B v}), \sin (\mathbf{B v})] \quad \mathbf{B} \sim \mathcal{N}\left(0, \sigma^{2}\right)\), too big mapping bandwith  \(\sigma\) leads to noisy images (overfit).
